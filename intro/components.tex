\secrel{Компоненты CMSIS}

\fig{intro/Overview.png}{width=\textwidth}

\begin{description}

\item[\term{CMSIS-Core} (\cm{}) \ref{Core}]: API для процессорных ядер \cm{}\ и
периферийных устройств. Он обес\-пе\-чи\-ва\-ет стандартизованный интерфейс для
\cm{0}, \cm{0+}, \cm{3}, \cm{4}, \cm{7}, \cm{23}, \cm{33},
SC000 и SC300. Также включены интринсик-функции команд SIMD для \cm{4},
\cm{7} и \cm{33} SIMD.

\item[\term{CMSIS-Core} (\ca{}) \ref{Core_A}]: API и базовая система времени
выполнения для ядра процессора и периферий\-ных устройств \ca{5}/A7/A9.

\item[\term{CMSIS-Driver} \ref{Driver}]: определяет обобщенные интерфейсы
драйверов периферии для промежуточного программ\-ного обеспечения, что делает
его повторно используемым для всех поддерживаемых устройств. API является
независимым от ОСРВ и обеспечивает интерфейс периферийных устройств
микроконтроллеров с middleware, которое например реализует стеки
коммуникационных протоколов, файловые системы или графические пользовательские
интерфейсы.

\item[\term{CMSIS-DSP} \ref{DSP}]: Коллекция библиотек ЦОС с более чем 60
функциями для различных типов данных: фиксированная точка (дроби q7, q15, q31) и
плавающая точка одинарной точности (32-битная). Библио\-те\-ка доступна для всех
ядер \cm{}. Реализации, оптимизированные для набора команд SIMD, доступ\-ны для
\cm{4}, \cm{7} и \cm{33}.

\item[\term{CMSIS-NN} \ref{NN}]: коллекция эффективных ядер нейронных сетей,
разработанных для максимальной произво\-ди\-тель\-ности и минимизации объема
памяти реализации приложений AI на процессорных ядрах \cm{}.

\item[\term{CMSIS-RTOS} v1 \ref{RTOS}]: Общий API для операционных систем
реального времени в виде эталонной реализации на основе RTX. Он обеспечивает
стандартизованный программный интерфейс совместимый со многими ОСРВ и
обеспечивает программные компоненты способные работать унифицированно на разных
RTOS.

\item[\term{CMSIS-RTOS} v2 \ref{RTOS2}]: расширяет CMSIS-RTOS v1 поддержкой
архитектуры Armv8-M, созданием динамичес\-ких объектов, поддержкой многоядерных
систем и бинарной совместимостью между различными ком\-пи\-ляторами с общим ABI.

\item[\term{CMSIS-Pack} \ref{Pack}]: описывает с помощью файла описания пакета
(PDSC) на основе XML параметры коллекции файлов\note{называемой
\term{программным пакетом} = проектом}, относящиеся к пользователю и устройству,
включая файлы исходных кодов, заголовков и библиотек, документацию, алгоритмы
программирования Flash-памяти, шаблоны исходных кодов и при\-меры проектов.
Средства разработки и веб-платформы используют файл PDSC\note{Дополнительную
информацию о содержимом программного пакета см. ARM::CMSIS Pack \ref{CM_Pack}}
для получения пара\-метров микроконтроллера, компонентов программного
обеспечения и конфигурации отладочной платы.

\item[\term{CMSIS-SVD} \ref{SVD}]: 
Описание системного представления\note{System View Description}\ для периферии.
Описывает периферийные устрой\-ства в формате XML-файла, и может использоваться
в качестве источника информации для отладчиков или заголовочных файлов, содержащих
определения периферийных регистров и прерываний.

\item[\term{CMSIS-DAP} \ref{DAP}]:
Отладочный порт доступа\note{Debug Access Port}. Стандартизованная прошивка для
модуля отладки, который подключается к порту отладчика CoreSight Debug Access
Port. CMSIS-DAP распространяется как отдель\-ный пакет и хорошо подходит для
интеграции в оценочные платы. Этот компонент загружается отдельно.

\item[\term{CMSIS-Zone} \ref{Zone}]: Определение и разбиение системных ресурсов.
Определяет методы описания системных ресурсов и разделения этих ресурсов на
несколько проектов и областей выполнения.

\end{description}

