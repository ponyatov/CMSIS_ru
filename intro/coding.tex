\pagebreak
\secrel{Правила кодирования}

\noindent
CMSIS использует следующие обязательные правила кодирования и соглашения:
\begin{itemize}[nosep]
  \item 
Код соответствует стандартам \anC\ (C99) и \cpp\ (C++03)
  \item 
Использует стандартные типы данных \anC, определенные в \verb|<stdint.h>|
  \item 
Переменные и параметры имеют полный тип данных
  \item 
Выражения для \verb|#define| констант заключены в круглые скобки 
  \item
Соответствует MISRA 2012 (но не требует его соблюдения). Нарушения правил
MISRA задокументированы.
\end{itemize}

\noindent
Кроме того, CMSIS рекомендует следующие соглашения для идентификаторов:

\begin{itemize}[nosep]
  \item 
CAPITAL имена для определения регистров ядра, регистров периферии и команд процессора
  \item 
CamelCase имена для определения имен функций и обработчиков прерываний
  \item 
\verb|Namespace_| префиксы пространства имен предотвращают конфликты с
идентификаторами пользователей и предоставляют функциональные группы (например,
для периферийных устройств, RTOS или библиотеки ЦОС)
\end{itemize}
 
\noindent
CMSIS документирован на уровне исходного кода с помощью:

\begin{itemize}[nosep]
  \item 
Комментарии в стиле \purec/\cpp. 
  \item 
Комментарии для функций в формате Doxygen, которые обеспечивают:
\begin{itemize}[nosep]
  \item 
краткое описание функции
  \item 
детальное описание функции
  \item 
детальное описание параметров
  \item 
детальная информация о возвращаемом значении 
\end{itemize}
\end{itemize}

\noindent
Пример документации Doxygen:

\begin{lstlisting}[language=C]
/** 
 * @brief  Enable Interrupt in NVIC Interrupt Controller
 * @param  IRQn  interrupt number that specifies the interrupt
 * @return none.
 * Enable the specified interrupt in the NVIC Interrupt Controller. */
\end{lstlisting}
%  * Other settings of the interrupt such as priority are not affected.
