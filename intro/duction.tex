\secrel{Введение}\secdown

\emph{Стандартный Интерфейс Программного Обеспечения Микроконтроллеров Cortex}
(\term{CMSIS}) является незави\-си\-мым от поставщика аппаратным уровнем
абстракции (HAL) для серии процессоров \cm{\rcirc}, и определяет общие
инструментальные интерфейсы. CMSIS обеспечивает постоянную поддержку устройств и
простые программные интерфейсы для процессора и периферийных устройств, упрощает
повторное использование программного обеспечения, уменьшает кривую обучения для
разработчиков микроконтроллеров, и сокращает время выхода новых устройств на
рынок.

Стандарт CMSIS формируется в тесном сотрудничестве с различными поставщиками
микросхем и программ\-но\-го обеспечения, и обеспечивает общий подход к
интерфейсу для периферийных устройств, операционных систем реального времени и
компонентов middleware\note{промежуточное программное обеспечение: HAL, 
реализации аппаратных стеков и сетевых протоколов, типовые библиотеки}.
CMSIS предназначен для обеспечения совместимости программных компонентов от нескольких
поставщиков middleware.

CMSIS версии 5 поддерживает также архитектуру
\href{http://www.arm.com/products/processors/instruction-set-architectures/armv8-m-architecture.php}{Armv8-M},
включая аппаратные расширения средств безопасности
\href{http://www.arm.com/products/processors/technologies/trustzone/index.php}{TrustZone\rcirc\
Armv8-M}, и процессоры \cm{23} и \cm{33}.

\secrel{Компоненты CMSIS}

\fig{intro/Overview.png}{width=\textwidth}

\begin{description}

\item[\term{CMSIS-Core} (\cm{}) \ref{Core}]: API для процессорных ядер \cm{}\ и
периферийных устройств. Он обес\-пе\-чи\-ва\-ет стандартизованный интерфейс для
\cm{0}, \cm{0+}, \cm{3}, \cm{4}, \cm{7}, \cm{23}, \cm{33},
SC000 и SC300. Также включены интринсик-функции команд SIMD для \cm{4},
\cm{7} и \cm{33} SIMD.

\item[\term{CMSIS-Core} (\ca{}) \ref{Core_A}]: API и базовая система времени
выполнения для ядра процессора и периферий\-ных устройств \ca{5}/A7/A9.

\item[\term{CMSIS-Driver} \ref{Driver}]: определяет обобщенные интерфейсы
драйверов периферии для промежуточного программ\-ного обеспечения, что делает
его повторно используемым для всех поддерживаемых устройств. API является
независимым от ОСРВ и обеспечивает интерфейс периферийных устройств
микроконтроллеров с middleware, которое например реализует стеки
коммуникационных протоколов, файловые системы или графические пользовательские
интерфейсы.

\item[\term{CMSIS-DSP} \ref{DSP}]: Коллекция библиотек ЦОС с более чем 60
функциями для различных типов данных: фиксированная точка (дроби q7, q15, q31) и
плавающая точка одинарной точности (32-битная). Библио\-те\-ка доступна для всех
ядер \cm{}. Реализации, оптимизированные для набора команд SIMD, доступ\-ны для
\cm{4}, \cm{7} и \cm{33}.

\item[\term{CMSIS-NN} \ref{NN}]: коллекция эффективных ядер нейронных сетей,
разработанных для максимальной произво\-ди\-тель\-ности и минимизации объема
памяти реализации приложений AI на процессорных ядрах \cm{}.

\item[\term{CMSIS-RTOS} v1 \ref{RTOS}]: Общий API для операционных систем
реального времени в виде эталонной реализации на основе RTX. Он обеспечивает
стандартизованный программный интерфейс совместимый со многими ОСРВ и
обеспечивает программные компоненты способные работать унифицированно на разных
RTOS.

\item[\term{CMSIS-RTOS} v2 \ref{RTOS2}]: расширяет CMSIS-RTOS v1 поддержкой
архитектуры Armv8-M, созданием динамичес\-ких объектов, поддержкой многоядерных
систем и бинарной совместимостью между различными ком\-пи\-ляторами с общим ABI.

\item[\term{CMSIS-Pack} \ref{Pack}]: описывает с помощью файла описания пакета
(PDSC) на основе XML параметры коллекции файлов\note{называемой
\term{программным пакетом} = проектом}, относящиеся к пользователю и устройству,
включая файлы исходных кодов, заголовков и библиотек, документацию, алгоритмы
программирования Flash-памяти, шаблоны исходных кодов и при\-меры проектов.
Средства разработки и веб-платформы используют файл PDSC\note{Дополнительную
информацию о содержимом программного пакета см. ARM::CMSIS Pack \ref{CM_Pack}}
для получения пара\-метров микроконтроллера, компонентов программного
обеспечения и конфигурации отладочной платы.

\item[\term{CMSIS-SVD} \ref{SVD}]: 
Описание системного представления\note{System View Description}\ для периферии.
Описывает периферийные устрой\-ства в формате XML-файла, и может использоваться
в качестве источника информации для отладчиков или заголовочных файлов, содержащих
определения периферийных регистров и прерываний.

\item[\term{CMSIS-DAP} \ref{DAP}]:
Отладочный порт доступа\note{Debug Access Port}. Стандартизованная прошивка для
модуля отладки, который подключается к порту отладчика CoreSight Debug Access
Port. CMSIS-DAP распространяется как отдель\-ный пакет и хорошо подходит для
интеграции в оценочные платы. Этот компонент загружается отдельно.

\item[\term{CMSIS-Zone} \ref{Zone}]: Определение и разбиение системных ресурсов.
Определяет методы описания системных ресурсов и разделения этих ресурсов на
несколько проектов и областей выполнения.

\end{description}


\clearpage
\secrel{Мотивация}

\emph{CMSIS} был создан, чтобы помочь стандартизации в отрасли. Он
\emph{обеспечивает} согласованные слои программ\-но\-го обеспечения и
\emph{поддержку устройств в широком диапазоне инструментов разработки и
микроконтроллеров} от разных производителей.
CMSIS это небольшой программный уровень, добавляющий накладные расходы, и при
этом не определяющий стандартные периферийные устройства\note{см. Standard
Peripherial Library (SPL) от конкретного производителя МК}.
Тем не менее, благодаря ему электронная промышленность может поддерживать
широкие вариации устройств на базе процессоров \cm{}, покры\-вае\-мые общим
стандартом.

\medskip
В частности, преимуществами CMSIS являются:
\medskip
\begin{itemize}[nosep]

\item В целом CMSIS снижает кривую обучения, затраты на разработку и время
выхода на рынок. Разработчики могут писать программное обеспечение быстрее,
используя множество простых в использовании стандар\-тизированных программных
интерфейсов.

\item Согласованные программные интерфейсы улучшают переносимость программного
обеспечения и его повторное использование. Универсальные программные библиотеки
и интерфейсы обеспечивают согласо\-ван\-ную программную среду.

\item Обеспечивает отладочный интерфейс, унифицированное представление
периферии, развертывание ПО, и поддержку устройств, при сокращении времени
выхода на рынок для устройств на новом микроконтроллере.

\item Предоставляет независимый от компилятора слой, который позволяет
использовать \textit{разные} компиляторы. CMSIS поддерживается широко
используемыми компиляторами.

\item Улучшает отладку программ с информацией о периферии для отладчиков,
ITM-каналами с форма\-ти\-ро\-ван\-ным выводом через printf, и поддержкой ядра
RTOS.

\item CMSIS поставляется в формате CMSIS-Pack, который обеспечивает быстрое
развертывание программного обеспечения, упрощает обновления, и обеспечивает
сквозную интеграцию с инструментами разработки.

\item CMSIS-Zone упрощает управление системными ресурсами и разделение,
поскольку он управляет конфи\-гу\-ра\-цией для нескольких процессоров, областей
памяти и периферийных устройств.

\end{itemize}
\pagebreak
\secrel{Правила кодирования}

\noindent
CMSIS использует следующие обязательные правила кодирования и соглашения:
\begin{itemize}[nosep]
  \item 
Код соответствует стандартам \anC\ (C99) и \cpp\ (C++03)
  \item 
Использует стандартные типы данных \anC, определенные в \verb|<stdint.h>|
  \item 
Переменные и параметры имеют полный тип данных
  \item 
Выражения для \verb|#define| констант заключены в круглые скобки 
  \item
Соответствует MISRA 2012 (но не требует его соблюдения). Нарушения правил
MISRA задокументированы.
\end{itemize}

\noindent
Кроме того, CMSIS рекомендует следующие соглашения для идентификаторов:

\begin{itemize}[nosep]
  \item 
CAPITAL имена для определения регистров ядра, регистров периферии и команд процессора
  \item 
CamelCase имена для определения имен функций и обработчиков прерываний
  \item 
\verb|Namespace_| префиксы пространства имен предотвращают конфликты с
идентификаторами пользователей и предоставляют функциональные группы (например,
для периферийных устройств, RTOS или библиотеки ЦОС)
\end{itemize}
 
\noindent
CMSIS документирован на уровне исходного кода с помощью:

\begin{itemize}[nosep]
  \item 
Комментарии в стиле \purec/\cpp. 
  \item 
Комментарии для функций в формате Doxygen, которые обеспечивают:
\begin{itemize}[nosep]
  \item 
краткое описание функции
  \item 
детальное описание функции
  \item 
детальное описание параметров
  \item 
детальная информация о возвращаемом значении 
\end{itemize}
\end{itemize}

\noindent
Пример документации Doxygen:

\begin{lstlisting}[language=C]
/** 
 * @brief  Enable Interrupt in NVIC Interrupt Controller
 * @param  IRQn  interrupt number that specifies the interrupt
 * @return none.
 * Enable the specified interrupt in the NVIC Interrupt Controller. */
\end{lstlisting}
%  * Other settings of the interrupt such as priority are not affected.


\secup