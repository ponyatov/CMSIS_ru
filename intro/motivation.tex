\clearpage
\secrel{Мотивация}

\emph{CMSIS} был создан, чтобы помочь стандартизации в отрасли. Он
\emph{обеспечивает} согласованные слои программ\-но\-го обеспечения и
\emph{поддержку устройств в широком диапазоне инструментов разработки и
микроконтроллеров} от разных производителей.
CMSIS это небольшой программный уровень, добавляющий накладные расходы, и при
этом не определяющий стандартные периферийные устройства\note{см. Standard
Peripherial Library (SPL) от конкретного производителя МК}.
Тем не менее, благодаря ему электронная промышленность может поддерживать
широкие вариации устройств на базе процессоров \cm{}, покры\-вае\-мые общим
стандартом.

\medskip
В частности, преимуществами CMSIS являются:
\medskip
\begin{itemize}[nosep]

\item В целом CMSIS снижает кривую обучения, затраты на разработку и время
выхода на рынок. Разработчики могут писать программное обеспечение быстрее,
используя множество простых в использовании стандар\-тизированных программных
интерфейсов.

\item Согласованные программные интерфейсы улучшают переносимость программного
обеспечения и его повторное использование. Универсальные программные библиотеки
и интерфейсы обеспечивают согласо\-ван\-ную программную среду.

\item Обеспечивает отладочный интерфейс, унифицированное представление
периферии, развертывание ПО, и поддержку устройств, при сокращении времени
выхода на рынок для устройств на новом микроконтроллере.

\item Предоставляет независимый от компилятора слой, который позволяет
использовать \textit{разные} компиляторы. CMSIS поддерживается широко
используемыми компиляторами.

\item Улучшает отладку программ с информацией о периферии для отладчиков,
ITM-каналами с форма\-ти\-ро\-ван\-ным выводом через printf, и поддержкой ядра
RTOS.

\item CMSIS поставляется в формате CMSIS-Pack, который обеспечивает быстрое
развертывание программного обеспечения, упрощает обновления, и обеспечивает
сквозную интеграцию с инструментами разработки.

\item CMSIS-Zone упрощает управление системными ресурсами и разделение,
поскольку он управляет конфи\-гу\-ра\-цией для нескольких процессоров, областей
памяти и периферийных устройств.

\end{itemize}