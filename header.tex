\documentclass[12pt,oneside]{book}

% \usepackage[paperwidth=118.8mm,paperheight=68.2mm,margin=2mm]{geometry}
\usepackage[paperwidth=210mm,paperheight=148.5mm,margin=5mm]{geometry}
%% font setup for screen reading
\renewcommand{\familydefault}{\sfdefault}\normalfont

%% hyperlinks pdf style
\usepackage[unicode,colorlinks=true]{hyperref}
%% colors
\usepackage{xcolor}
\definecolor{red}{rgb}{0.5,0,0}
\definecolor{green}{rgb}{0,0.3,0}
\definecolor{blue}{rgb}{0,0,0.7}

% graphics
\usepackage[pdftex]{graphicx}
\newcommand{\fig}[2]{\noindent\includegraphics[#2]{#1}}
% \bigskip\noindent\includegraphics[#3]{#2}\textbf{#1}\bigskip}

% hyphenation and i18n
\usepackage[T1,T2A]{fontenc}
\usepackage[utf8]{inputenc}
\usepackage[russian,english]{babel}

% relative sectioning
\usepackage{ifthen}
\newcounter{secdepth}\setcounter{secdepth}{0}
\newcommand{\secup}{\addtocounter{secdepth}{1}}
\newcommand{\secdown}{\addtocounter{secdepth}{-1}}
\newcommand{\secrel}[1]{
\ifthenelse{\equal{\value{secdepth}}{0}}{\part{#1}}{}
\ifthenelse{\equal{\value{secdepth}}{-1}}{\chapter{#1}}{}
\ifthenelse{\equal{\value{secdepth}}{-2}}{\section{#1}}{}
\ifthenelse{\equal{\value{secdepth}}{-3}}{\subsection{#1}}{}
\ifthenelse{\equal{\value{secdepth}}{-4}}{\subsubsection{#1}}{}
\ifthenelse{\equal{\value{secdepth}}{-5}}{\paragraph{#1}\ \\}{}
}
\newcommand{\secly}[1]{
\section*{#1}
\addcontentsline{toc}{subsection}{#1}
}

% misc
\newcommand{\email}[1]{$<$\href{mailto:#1}{#1}$>$}
\newcommand{\note}[1]{\,\footnote{\ #1}}
\renewcommand{\emph}[1]{\textcolor{blue}{#1}}
\newcommand{\term}[1]{\textcolor{green}{#1}}
%% [nosep] option in lists/enums
\usepackage{enumitem}
%% frame box
\usepackage{framed}

\newcommand{\rcirc}{\textsuperscript{\textregistered}}
\newcommand{\cm}[1]{Cortex-M#1}
\newcommand{\ca}[1]{Cortex-A#1}

\newcommand{\pg}{\pagebreak\noindent}

%% listings
\usepackage{listings}
\lstset{
basicstyle=\small,
% % frame=single,
tabsize=4,
% keywordstyle=\color{green},
% commentstyle=\color{blue},
% stringstyle=\color{red}
}

\newcommand{\cpp}{$C^{++}$}
\newcommand{\purec}{$C$}
\newcommand{\anC}{$C_{ANSI}$}
