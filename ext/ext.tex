\pagebreak
\secrel{Дополнительные материалы}\secdown

В состав проекта перевода включено краткое руководство по применению
микроконтроллеров на ядре \cm{}\ для разработки ПО для встраиваемых систем
на Linux с применением GNU compiler toolchain.

\bigskip\noindent
Также в проект добавлен минимальный набор файлов CMSIS Standart Peripherial
Library для нескольких распространненных микроконтроллеров и модулей, с которых
можно начать освоение разработки ПО для \cm{}.

\bigskip
\begin{tabular}{l ll}
src/ && исходные коды примеров \\
src/ & STM32/ & файлы STM32 Std Prerih Library / Cube \\
lib/ && каталог сборки (статических) библиотек \\
ld/ && скрипты GNU linker \ref{ld} \\
\end{tabular}

\pagebreak
\secrel{GNU toolchain: arm-none-eabi}\label{gnu}\secdown

\begin{itemize}[nosep]
  \item мультиплатформенно (Win/Linux/MacOS, x86 PC, Raspberry Pi)
  \item бесплатно в т.ч. для коммерческого использования
  \item большая база пользоватей: много информации по использованию и
  сложностям на форумах
  \item активно развивается OpenSource community: отсутствие vendor lock на
  разработчика компилятора
\end{itemize}

\noindent
\fig{tmp/toolchain.png}{width=\textwidth}

\secrel{Установка на Debian GNU/Linux}\secdown

\secrel{Готовые пакеты из репозитория дистрибутива}

\begin{lstlisting}
$ sudo apt install gcc-arm-none-eabi gcc-arm-none-eabi qemu-system-arm git make
\end{lstlisting}

\secrel{texane: утилита для программатора ST/Link}

\begin{framed}
\noindent не обновляйте прошивку ST/Link\ --- перестанет работать с
texane/st-util
\end{framed}

\begin{lstlisting}
$ wget -c https://github.com/texane/stlink/archive/1.5.0.tar.gz
$ tar zx < 1.5.0.tar.gz | cd texane-1.5.0
$ ./configure [--prefix=/usr/local] && make && sudo make install
\end{lstlisting}

\secup

\pagebreak
\secrel{Makefile скрипты сборки}\secdown

Проект организован в виде нескольких каталогов, и файлов Makefile: скриптов для
типовой утилиты \verb|make|.

\noindent
Чаще всего используются выражения с переменными:

\begin{tabular}{l l}
\verb|VAR  = value| & задание значения переменной \\
\verb|VAR ?= value| & задание если переменная ранее не определена \\
\verb|VAR += value| & добавить значение \\ 
\verb|PATH = $(CURDIR)/bin:$(PATH)| & \verb|(...)| подстановка значения
переменной\\
\verb|$(CURDIR)| & \term{встроенная переменная} GNU make:\\&полный путь к
текущему каталогу,\\&типа \verb|/home/user/acsip/lib|\\
\verb|$(PATH)| & \term{системная переменная}\\&список каталогов, к которых будет
искаться любая программа,\\&указанная по имени из командной строки или Makefile/скрипта\\
\end{tabular}

\secrel{/mk.mk общая часть используемая через include}

\begin{itemize}[nosep]
  \item \verb|TARGET=arm-none-eabi| \term{триплет} целевой платформы (ARM без
  Linux и libc)
  \item утилиты GNU toolchain
  \item \verb|CFLAGS +=| флаги для компилятора \purec
\end{itemize}

\lstinputlisting{mk.mk}

\secup
\secrel{Компиляция статических библиотек}

В каталоге \verb|/acsip/lib/| находятся прекомпилированные объектные файлы и
библиотеки:

\begin{tabular}{l l}

\verb|startup_*.o| & стартовый код для каждого поддерживаемого МК:\\
& таблица прерываний и обработчик \verb|Reset_Handler|\\

\verb|lib*.lib| & прекомпилированная Std Periph Library \\
\verb|libCMSIS.lib| & общий код CMSIS \\

\end{tabular}


\secup
\secrel{Скрипты линкера (задание карты памяти МК)}\label{ld}


\clearpage
\secrel{Отладка и прошивка МК под Linux}\secdown

\secrel{Запуск texane/st-util как gdb-server}

Подключите ST/Link и запустите texane/st-util как \term{gdb-server}\note{для
прошивки микроконтроллера используется режим \term{удаленной отладки}}:
\begin{lstlisting}
dpon@dpon:~/LoRaMac-node$ st-util 
st-util 1.5.0
2018-04-26T15:07:54 INFO usb.c: -- exit_dfu_mode
2018-04-26T15:07:54 INFO common.c: Loading device parameters....
2018-04-26T15:07:54 INFO common.c: Device connected is: F0 device, id 0x20006440
2018-04-26T15:07:54 INFO common.c: SRAM size: 0x2000 bytes (8 KiB),
							Flash: 0x10000 bytes (64 KiB) in pages of 1024 bytes
2018-04-26T15:07:54 INFO gdb-server.c: Chip ID is 00000440, Core ID is  0bb11477.
2018-04-26T15:07:54 INFO gdb-server.c: Listening at *:4242...
\end{lstlisting}

\secrel{Запуск отладчика}

\begin{lstlisting}
dpon@dpon:~/acsip/src$ arm-none-eabi-gdb -ex "target remote :4242" hello.elf 
GNU gdb (7.12-6+9+b2) 7.12.0.20161007-git
Copyright (C) 2016 Free Software Foundation, Inc.
License GPLv3+: GNU GPL version 3 or later <http://gnu.org/licenses/gpl.html>
This is free software: you are free to change and redistribute it.
There is NO WARRANTY, to the extent permitted by law.  Type "show copying"
and "show warranty" for details.
This GDB was configured as "--host=x86_64-linux-gnu --target=arm-none-eabi".
Type "show configuration" for configuration details.
For bug reporting instructions, please see:
<http://www.gnu.org/software/gdb/bugs/>.
Find the GDB manual and other documentation resources online at:
<http://www.gnu.org/software/gdb/documentation/>.
For help, type "help".
Type "apropos word" to search for commands related to "word"...
Reading symbols from hello.elf...done.
Remote debugging using :4242
0x08000238 in ?? ()
(gdb) 
\end{lstlisting}

\secrel{Файл инициализации отладчика .gdb}

При запуске отладчика можно указать файл инициализации, в котором прописать
команды, выполняемые при старте отладки:
\lstinputlisting{src/ram.gdb}

\secrel{Графическая оболочка отладчика ddd}

Для удобства можно использовать графическую оболочку ddd:

\begin{lstlisting}{title=Makefile}
$ ddd --debugger "$(GDB) -x ram.gdb $<"
\end{lstlisting}

\clearpage
\fig{ext/ddd.jpg}{height=.45\textheight}

\medskip
Главная фича ddd: он умеет показывать структуры данных графически:
\medskip

\fig{ext/ddd.png}{height=.45\textheight}

\secup
\pagebreak
\secrel{Эмулятор QEMU-arm: микроконтроллеры \cm{3}}\secdown
\secrel{Stellaris LM3S811}

\lstinputlisting{ld/LM3S811.ld}
\secup

\pagebreak
\secrel{Микроконтроллеры STM32}\secdown
\secrel{stm32f0discovery: STM32F051}

\begin{tabular}{l l}

\mbox{\fig{ext/f0discovery.jpg}{width=.5\textwidth}}

&

\bigskip
\begin{tabular}{l l}
CPU:&STM32F051R8T6\\
Flash ROM:&64K\\
SRAM:&8K\\
\end{tabular}

\\
\end{tabular}


\lstinputlisting{ld/STM32F051R8T6.ld}

\pagebreak
\secrel{Модуль LoRaWAN AcSIP S76: \cm{3} + SX1276 RF}

\begin{tabular}{l l}

\end{tabular}
\secup


\pagebreak
\secrel{Миландр 1986ВЕ9х \cm{3}}

\begin{tabular}{l l}

\mbox{\fig{ext/1986VE.png}{height=0.5\textheight}}

&

\begin{tabular}{l l l l}
& Flash ROM & SRAM & частота \\
&&& ядра \\
\hline
1986ВЕ1Т & 128K & 49K & 144 MHz\\
1986ВЕ92 & & \\
\end{tabular}

\\

\end{tabular}


\secup
