\pagebreak
\secrel{Дополнительные материалы}\secdown

В состав проекта перевода включено краткое руководство по применению
микроконтроллеров на ядре \cm{}\ для разработки ПО для встраиваемых систем
на Linux с применением GNU compiler toolchain.

\bigskip\noindent
Также в проект добавлен минимальный набор файлов CMSIS Standart Peripherial
Library для нескольких распространненных микроконтроллеров и модулей, с которых
можно начать освоение разработки ПО для \cm{}.

\bigskip
\begin{tabular}{l ll}
src/ && исходные коды примеров \\
src/ & STM32/ & файлы STM32 Std Prerih Library / Cube \\
lib/ && каталог сборки (статических) библиотек \\
ld/ && скрипты GNU linker \ref{ld} \\
\end{tabular}

\pagebreak
\secrel{GNU toolchain: arm-none-eabi}\label{gnu}\secdown

\begin{itemize}[nosep]
  \item мультиплатформенно (Win/Linux/MacOS, x86 PC, Raspberry Pi)
  \item бесплатно в т.ч. для коммерческого использования
  \item большая база пользоватей: много информации по использованию и
  сложностям на форумах
  \item активно развивается OpenSource community: отсутствие vendor lock на
  разработчика компилятора
\end{itemize}

\noindent
\fig{tmp/toolchain.png}{width=\textwidth}

\secrel{Установка на Debian GNU/Linux}\secdown

\secrel{Готовые пакеты из репозитория дистрибутива}

\begin{lstlisting}
$ sudo apt install gcc-arm-none-eabi gcc-arm-none-eabi qemu-system-arm git make
\end{lstlisting}

\secrel{texane: утилита для программатора ST/Link}

\begin{framed}
\noindent не обновляйте прошивку ST/Link\ --- перестанет работать с
texane/st-util
\end{framed}

\begin{lstlisting}
$ wget -c https://github.com/texane/stlink/archive/1.5.0.tar.gz
$ tar zx < 1.5.0.tar.gz | cd texane-1.5.0
$ ./configure [--prefix=/usr/local] && make && sudo make install
\end{lstlisting}

\secup

\pagebreak
\secrel{Makefile скрипты сборки}\secdown

Проект организован в виде нескольких каталогов, и файлов Makefile: скриптов для
типовой утилиты \verb|make|.

\noindent
Чаще всего используются выражения с переменными:

\begin{tabular}{l l}
\verb|VAR  = value| & задание значения переменной \\
\verb|VAR ?= value| & задание если переменная ранее не определена \\
\verb|VAR += value| & добавить значение \\ 
\verb|PATH = $(CURDIR)/bin:$(PATH)| & \verb|(...)| подстановка значения
переменной\\
\verb|$(CURDIR)| & \term{встроенная переменная} GNU make:\\&полный путь к
текущему каталогу,\\&типа \verb|/home/user/acsip/lib|\\
\verb|$(PATH)| & \term{системная переменная}\\&список каталогов, к которых будет
искаться любая программа,\\&указанная по имени из командной строки или Makefile/скрипта\\
\end{tabular}

\secrel{/mk.mk общая часть используемая через include}

\begin{itemize}[nosep]
  \item \verb|TARGET=arm-none-eabi| \term{триплет} целевой платформы (ARM без
  Linux и libc)
  \item утилиты GNU toolchain
  \item \verb|CFLAGS +=| флаги для компилятора \purec
\end{itemize}

\lstinputlisting{mk.mk}

\secup
\secrel{Компиляция статических библиотек}

В каталоге \verb|/acsip/lib/| находятся прекомпилированные объектные файлы и
библиотеки:

\begin{tabular}{l l}

\verb|startup_*.o| & стартовый код для каждого поддерживаемого МК:\\
& таблица прерываний и обработчик \verb|Reset_Handler|\\

\verb|lib*.lib| & прекомпилированная Std Periph Library \\
\verb|libCMSIS.lib| & общий код CMSIS \\

\end{tabular}


\secup
\secrel{Скрипты линкера (задание карты памяти МК)}\label{ld}


\pagebreak
\secrel{Эмулятор QEMU-arm: микроконтроллеры \cm{3}}\secdown
\secrel{Stellaris LM3S811}

\lstinputlisting{ld/LM3S811.ld}
\secup

\pagebreak
\secrel{Микроконтроллеры STM32}\secdown
\secrel{stm32f0discovery: STM32F051}

\begin{tabular}{l l}

\mbox{\fig{ext/f0discovery.jpg}{width=.5\textwidth}}

&

\bigskip
\begin{tabular}{l l}
CPU:&STM32F051R8T6\\
Flash ROM:&64K\\
SRAM:&8K\\
\end{tabular}

\\
\end{tabular}


\lstinputlisting{ld/STM32F051R8T6.ld}

\pagebreak
\secrel{Модуль LoRaWAN AcSIP S76: \cm{3} + SX1276 RF}

\begin{tabular}{l l}

\end{tabular}
\secup


\pagebreak
\secrel{Миландр 1986ВЕ9х \cm{3}}

\begin{tabular}{l l}

\mbox{\fig{ext/1986VE.png}{height=0.5\textheight}}

&

\begin{tabular}{l l l l}
& Flash ROM & SRAM & частота \\
&&& ядра \\
\hline
1986ВЕ1Т & 128K & 49K & 144 MHz\\
1986ВЕ92 & & \\
\end{tabular}

\\

\end{tabular}


\secup
