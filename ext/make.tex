\pagebreak
\secrel{Makefile скрипты сборки}\secdown

Проект организован в виде нескольких каталогов, и файлов Makefile: скриптов для
типовой утилиты \verb|make|.

\noindent
Чаще всего используются выражения с переменными:

\begin{tabular}{l l}
\verb|VAR  = value| & задание значения переменной \\
\verb|VAR ?= value| & задание если переменная ранее не определена \\
\verb|VAR += value| & добавить значение \\ 
\verb|PATH = $(CURDIR)/bin:$(PATH)| & \verb|(...)| подстановка значения
переменной\\
\verb|$(CURDIR)| & \term{встроенная переменная} GNU make:\\&полный путь к
текущему каталогу,\\&типа \verb|/home/user/acsip/lib|\\
\verb|$(PATH)| & \term{системная переменная}\\&список каталогов, к которых будет
искаться любая программа,\\&указанная по имени из командной строки или Makefile/скрипта\\
\verb|TODAY = $(shell date +%y%m%d)| & подставить результат выполнения команды\\
\end{tabular}

\secrel{/mk.mk общая часть используемая через include}

\begin{itemize}[nosep]
  \item \verb|TARGET=arm-none-eabi| \term{триплет} целевой платформы (ARM без
  Linux и libc)
  \item \verb|TOPDIR=| полный путь каталога проекта: \verb|/home/dpon/CMSIS_ru| 
  \item утилиты GNU toolchain
  \item \verb|CFLAGS +=| флаги для компилятора \purec
  \begin{itemize}[nosep]
    \item включение отладочной информации (имена функций, адреса меток,..)
    \item флаги оптимизации
    \item каталоги размещения заголовочных файлов и библиотек 
  \end{itemize}
\end{itemize}

\pg
\lstinputlisting{mk.mk}

\secup