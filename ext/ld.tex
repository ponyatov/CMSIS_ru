\pagebreak
\secrel{Скрипты линкера (задание карты памяти МК)}\label{ld}\secdown

Каждая \term{целевая система} имеет собственную карту памяти: отличаются размеры
Flash/RAM, размеры кучи и стека, адреса ввода/вывода устройств, и особенности
адресации. Для настройки карты памяти применяются \term{скрипты линкера}.

\secrel{Задание карты памяти целевой системы}

\lstinputlisting{ld/STM32F051R8T6.ld}

\bigskip\noindent
При линковке для всех CMSIS/\cm{}\ микроконтроллеров должен использоваться этот
общий файл:

\lstinputlisting[title=ld/ANY.ld]{ld/ANY.ld}

\pagebreak
\secrel{Секции объектных файлов}

При компиляции \emph{каждого модуля} программы в \term{объектном файле}
создаются следующие \term{сегменты памяти}:

\begin{lstlisting}
SECTIONS {
\end{lstlisting}

\begin{tabular}{l l}
\verb|.text| & исполняемый машинный код \\
\verb|.rodata| & константы \\
\verb|.data| & переменные и массивы с заданными начальными значениями \\
\verb|.bss| & динамическая памяти и неинициализированные массивы \\
\verb|.stack| & стек (распределяется загрузчиком или ОС) \\
\end{tabular}

\bigskip
\noindent Для микроконтроллеров добавляется \term{таблица векторов
прерываний}:\bigskip

\begin{tabular}{l l}
\verb|.isr_vector| & таблица прерываний \\
\end{tabular}

\bigskip
\noindent Стандартная библиотека языка \purec\ libc (newlib, uclibc,..)\\ и
стартовый код который добавляет компилятор (из файлов crt*.o и libgcc)\\
предполагает наличие кода инициализации и финализации\note{выполняется при
завершении программы}:\bigskip

\begin{tabular}{l l}
\verb|.init| &
\verb|.fini| \\
\verb|.preinit_array_*| & \\
\verb|.init_array*| &
\verb|.fini_array*| \\
\end{tabular}


\bigskip
\noindent И наконец \cpp\ добавляет \term{статические конструкторы и
деструкторы}:\bigskip

\begin{tabular}{l l}
\verb|.ctors| & конструкторы \\
\verb|.dtors| & деструкторы \\
\verb|.eh_frame*| & таблицы исключений \\
\end{tabular}

\secrel{Сборка загрузочного образа в формате ELF}

Утилита программатора требует на входе прошивку\ --- \emph{один}
\term{объектный файл} в формате ELF, или файл .hex который
создается из .elf с помощью утилиты \verb|arm-none-eabi-objcopy|.

\begin{lstlisting}
dpon@dpon:~/acsip/src$ arm-none-eabi-objdump -x hello.elf 

hello.elf:     file format elf32-littlearm
hello.elf
architecture: arm, flags 0x00000112:
EXEC_P, HAS_SYMS, D_PAGED
start address 0x200000c1

Program Header:
    LOAD off    0x00010000 vaddr 0x20000000 paddr 0x20000000 align 2**16
         filesz 0x000000fc memsz 0x000000fc flags r-x
private flags = 5000200: [Version5 EABI] [soft-float ABI]
\end{lstlisting}

\begin{tabular}{l p{.75\textwidth}}
\verb|elf32| & Executable and Linkable Format 32 bit, для 64-битных архитектур
есть elf64\\
\verb|littlearm| & Архитектура ARM \term{little-endian}\\&порядок байт в памяти
как у x86, \emph{по адресу 32-битного слова лежит младший байт}\\
&еще бывают big-endian процессоры с сетевым порядком байт (TCP/IP)\\
\verb|EXEC_P| & исполняемый файл, может быть запущен ОС напрямую или
прошит программатором\\
\verb|HAS_SYMS| & содержит \term{таблицу сиволов}: список всех функций и
переменных\\& с адресами их размещения (относительно начала сегментов)\\
\end{tabular}

\bigskip \noindent
Файл прошивки
должен содержать \emph{определенный набор сегментов}, привязанных на адреса,
специфичные для конкретного микроконтроллера:\\

\begin{lstlisting}
Sections:
Idx Name          Size      VMA       LMA       File off  Algn
  0 .isr_vector   000000c0  20000000  20000000  00010000  2**0
                  CONTENTS, ALLOC, LOAD, READONLY, DATA
  1 .init         00000000  200000c0  200000fc  000100fc  2**0
                  CONTENTS, ALLOC, LOAD, CODE
  2 .text         0000003c  200000c0  200000c0  000100c0  2**2
                  CONTENTS, ALLOC, LOAD, READONLY, CODE
  3 .data         00000000  200000fc  200000fc  000100fc  2**0
                  CONTENTS, ALLOC, LOAD, DATA
  4 .bss          00000000  200000fc  200000fc  00000000  2**0
                  ALLOC
  5 .debug_line   000000aa  00000000  00000000  000100fc  2**0
                  CONTENTS, READONLY, DEBUGGING
  6 .debug_info   000000d0  00000000  00000000  000101a6  2**0
                  CONTENTS, READONLY, DEBUGGING
  7 .debug_abbrev 00000054  00000000  00000000  00010276  2**0
                  CONTENTS, READONLY, DEBUGGING
  8 .debug_aranges 00000048  00000000  00000000  000102d0  2**3
                  CONTENTS, READONLY, DEBUGGING
  9 .debug_ranges 00000030  00000000  00000000  00010318  2**3
                  CONTENTS, READONLY, DEBUGGING
 10 .debug_str    000000f7  00000000  00000000  00010348  2**0
                  CONTENTS, READONLY, DEBUGGING
 11 .debug_frame  0000002c  00000000  00000000  00010440  2**2
                  CONTENTS, READONLY, DEBUGGING
SYMBOL TABLE:
...
\end{lstlisting}

\begin{tabular}{l p{.75\textwidth}}
\verb|size| & размер сегмента \emph{в байтах} \\
\verb|VMA| & Virtual Memory Address: адрес, по которому сегмент будет
размещен в памяти МК\\
& для STM32: \verb|RAM:0x2000000|, \verb|Flash:0x0B000000| отображен на
\verb|0x00000000|\\
\verb|LMA| & Load Memory Address: адрес загрузки (хранения): для некоторых МК
используется копирование кода из Flash в RAM, поэтому у сегментов с таким кодом
\verb|VMA->RAM| и \verb|LMA->FLASH|;
при запуске контроллера стартовый код выполняет копирование кода перед его
выполнением\\
\verb|start address| & точка входа программы (указывается для отладчика),
для микроконтроллера реальной точкой входа является обработчик прерывания
\verb|Reset_Handler|\\
\verb|ALLOC| & под сегмент будет выделяться память \\
\verb|LOAD| & сегмент будет загружаться (из файловой системы или
программатором)\\
\verb|CODE| & машинный код \\
\verb|DATA| & данные \\
\verb|READONLY| & неизменяемая область памяти \\
\verb|DEBUGGING| & данные для отладчика \\
\end{tabular}

\secrel{Линковка для прошивки во Flash}

Для получения прошивки все \term{объектные файлы} и \term{библитеки} нужно
обработать \term{линкером} \verb|arm-none-eabi-ld| (или \verb|gold|), определить
какие куски используются программой, или отмечены флагом \verb|KEEP|, и
\term{слинковать} файл прошивки только из \emph{используемых сегментов}.
По умолчанию такая оптимизирующая линковка выключена\note{сливаются все
указанные объектные файлы целиком, из библиотек берутся объектные файлы на
которые есть ссылки}, для использования LTO\note{Link Time Optimization}\ нужно
использовать дополнительные опции компилятора и линкера:
\begin{lstlisting}
CFLAGS += -ffunction-sections
LDFLAGS += --gc-collect
\end{lstlisting}

\secrel{Линковка для отладки в RAM}

\lstinputlisting[title=ld/RAM.ld]{ld/RAM.ld}

\secup
