\clearpage
\secrel{Отладка и прошивка МК под Linux}\secdown

\secrel{Запуск texane/st-util как gdb-server}

Подключите ST/Link и запустите texane/st-util как \term{gdb-server}\note{для
прошивки микроконтроллера используется режим \term{удаленной отладки}}:
\begin{lstlisting}
dpon@dpon:~/LoRaMac-node$ st-util 
st-util 1.5.0
2018-04-26T15:07:54 INFO usb.c: -- exit_dfu_mode
2018-04-26T15:07:54 INFO common.c: Loading device parameters....
2018-04-26T15:07:54 INFO common.c: Device connected is: F0 device, id 0x20006440
2018-04-26T15:07:54 INFO common.c: SRAM size: 0x2000 bytes (8 KiB),
							Flash: 0x10000 bytes (64 KiB) in pages of 1024 bytes
2018-04-26T15:07:54 INFO gdb-server.c: Chip ID is 00000440, Core ID is  0bb11477.
2018-04-26T15:07:54 INFO gdb-server.c: Listening at *:4242...
\end{lstlisting}

\secrel{Запуск отладчика}

\begin{lstlisting}
dpon@dpon:~/acsip/src$ arm-none-eabi-gdb -ex "target remote :4242" hello.elf 
GNU gdb (7.12-6+9+b2) 7.12.0.20161007-git
Copyright (C) 2016 Free Software Foundation, Inc.
License GPLv3+: GNU GPL version 3 or later <http://gnu.org/licenses/gpl.html>
This is free software: you are free to change and redistribute it.
There is NO WARRANTY, to the extent permitted by law.  Type "show copying"
and "show warranty" for details.
This GDB was configured as "--host=x86_64-linux-gnu --target=arm-none-eabi".
Type "show configuration" for configuration details.
For bug reporting instructions, please see:
<http://www.gnu.org/software/gdb/bugs/>.
Find the GDB manual and other documentation resources online at:
<http://www.gnu.org/software/gdb/documentation/>.
For help, type "help".
Type "apropos word" to search for commands related to "word"...
Reading symbols from hello.elf...done.
Remote debugging using :4242
0x08000238 in ?? ()
(gdb) 
\end{lstlisting}

\secrel{Файл инициализации отладчика .gdb}

При запуске отладчика можно указать файл инициализации, в котором прописать
команды, выполняемые при старте отладки:
\lstinputlisting{src/ram.gdb}

\secrel{Графическая оболочка отладчика ddd}

Для удобства можно использовать графическую оболочку ddd:

\begin{lstlisting}{title=Makefile}
$ ddd --debugger "$(GDB) -x ram.gdb $<"
\end{lstlisting}

\clearpage
\fig{ext/ddd.jpg}{height=.45\textheight}

\medskip
Главная фича ddd: он умеет показывать структуры данных графически:
\medskip

\fig{ext/ddd.png}{height=.45\textheight}

\secup