\pagebreak
\secrel{Обзор}

CMSIS-Core (\cm{}) реализует базовую систему времени выполнения для устройств на
процессорном ядре \cm{}, и предоставляет пользователю доступ к ядру и
периферийным устройствам. В деталях он определяет:

\begin{description}

\item[\term{Hardware Abstraction Layer} (\term{HAL})] Уровень абстракции
аппаратных средств для регистров процессора \cm{}\ со стандартизованными
определениями для регистров SysTick, NVIC, System Control Block, регистров MPU,
регистров FPU и функций доступа к ядру.

\item[Имена системных исключений] для взаимодействия с исключениями системы без
проблем совместимости.

Methods to organize header files] that makes it easy to learn new Cortex-M
microcontroller products and improve software portability. This includes naming
conventions for device-specific interrupts.

\item[Методы организации заголовочных файлов], которые позволяют легко освоить
новые продукт на базе микроконтроллера \cm{}, и улучшить переносимость
программного обеспечения.

\item[Имена прерываний] Соглашения об именах для прерываний,
специфичных для устройства.

\item[Методы инициализации системы], которые будут использоваться каждым
поставщиком MCU. Например, стандартизованная функция \verb|SystemInit()|
необходима для настройки тактирования.

\item[Интринсик-функции], используемые для генерации инструкций CPU, которые не
поддерживаются стан\-дарт\-ны\-ми функциями \purec.

\item[тактовая частота системы] переменная упрощающая настройку таймера
SysTick.

\end{description}

\pg
В следующих разделах представлена подробная информация о CMSIS-Core (\cm{}):

\begin{description}
\item[Использование CMSIS во встраиваемых приложениях] \ref{using_pg} 
описывает настройку проекта и показывает простой пример программы.

\item[Using TrustZone\rcirc\ for Armv8-M] описывает, как использовать расширения
безопасности, доступные в архитектуре Armv8-M.

\item[Шаблоны устройств CMSIS-Core Device] \ref{templates_pg}\ подробно
описывает файлы CMSIS-Core (\cm{}) и объясняет поставщикам электронных
устройства как адаптировать файлы шаблонов, предоставленные Arm.

\item[Отклонения от MISRA-C] \ref{coreMISRA}\ описывает нарушения стандарта
безопасного кода MISRA.

\item[Справочник] \ref{coreRef}\ подробно описывает особенности и функции
заголовочного файла устройства \verb|<device.h>|.

\item[Структуры данных] \ref{coreStruc}\ подробно описывает структуры данных
заголовочного файла устройства \verb|<device.h>|.

\end{description}